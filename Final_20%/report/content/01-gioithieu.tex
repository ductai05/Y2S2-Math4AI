\newpage
\section{Giới thiệu}

\paragraph{}{Đây là bài báo cáo cho \textbf{Final project: CLIP}, môn Phương pháp toán cho Trí tuệ nhân tạo, lớp Trí tuệ nhân tạo Khóa 2023 (23TNT1), Khoa Công nghệ thông tin, Trường Đại học Khoa học tự nhiên - Đại học Quốc gia TP.HCM. Trong bài báo cáo này, chúng tôi sẽ tóm tắt, giải thích về \textbf{CLIP} - một mô hình đa phương thức thu hẹp khoảng cách giữa thị giác máy tính và xử lý ngôn ngữ tự nhiên. Chúng tôi cũng nghiên cứu một số ứng dụng, phát triển của CLIP và đánh giá về mô hình này.}

\paragraph{}{\textbf{Báo cáo được thực hiện bởi nhóm 6\footnote{Nhóm 6, môn Phương pháp toán cho Trí tuệ nhân tạo, lớp Trí tuệ nhân tạo Khóa 2023 (HCMUS, VNUHCM)}, gồm các thành viên:}} 
\begin{itemize}
    \item Nguyễn Đình Hà Dương (23122002)
    \item Nguyễn Lê Hoàng Trung (23122004)
    \item Đinh Đức Tài (23122013)
    \item Hoàng Minh Trung (23122014)
\end{itemize}

\paragraph{}{\textbf{Đường dẫn repository Github của báo cáo:} \href{https://github.com/ductai05/Math-For-AI}{https://github.com/ductai05/Math-For-AI} \cite{repo}}

\paragraph{}{\textbf{Đường dẫn demo Youtube:}} \href{https://www.youtube.com/watch?v=1G227RKnv-k}{https://www.youtube.com/watch?v=1G227RKnv-k} \cite{demo}

\paragraph{}{\textbf{Bảng phân công nhiệm vụ cho từng thành viên:}}

\begin{table}[H]
\centering
\label{tab:nhiemvu}
\begin{tabular}{|c|c|l|}
\hline
\textbf{Họ và tên} &
  \textbf{MSSV} &
  \multicolumn{1}{c|}{\textbf{Nhiệm vụ}} \\ \hline
\begin{tabular}[c]{@{}c@{}}Nguyễn Đình\\ Hà Dương\end{tabular} &
  23122002 &
  \begin{tabular}[c]{@{}l@{}}- Ứng dụng CLIP trong truy vấn ảnh-văn bản. Thử nghiệm ViT.\\ - Ứng dụng nâng cao CLIP vào phân biệt, nhận diện khuôn mặt.\end{tabular} \\ \hline
\begin{tabular}[c]{@{}c@{}}Nguyễn Lê\\ Hoàng Trung\end{tabular} &
  23122004 &
  \begin{tabular}[c]{@{}l@{}}- Các kĩ thuật huấn luyện CLIP. Chỉnh sửa video thuyết trình.\\ - So sánh biến thể, ưu nhược điểm CLIP.\end{tabular} \\ \hline
\begin{tabular}[c]{@{}c@{}}Đinh \\ Đức Tài\end{tabular} &
  23122013 &
  \begin{tabular}[c]{@{}l@{}}- Trình bày các mô hình tương tự CLIP. Review report.\\ - Demo code CLIP, BLIP, ALIGN và so sánh.\end{tabular} \\ \hline
\begin{tabular}[c]{@{}c@{}}Hoàng\\ Minh Trung\end{tabular} &
  23122014 &
  \begin{tabular}[c]{@{}l@{}}- Các kĩ thuật nền tảng của CLIP.\\ - Tìm hiểu, trình bày WIT, ResNet, ViT, Transformer.\end{tabular} \\ \hline
\end{tabular}
\end{table}

\paragraph{}{\textbf{Các thư viện và công nghệ sử dụng:}}

\begin{itemize}
    \item Numpy, Pandas: thư viện Python để xử lý số học, thao tác và xử lý dữ liệu.
    \item transformers, torch, PIL: Các thư viện AI và xử lý hình ảnh.
    \item Jupyter Notebook (thông qua jupyter, ipykernel): Môi trường làm việc tương tác cho phép kết hợp mã thực thi, Markdown, công thức toán học và trực quan hóa.
    \item Git, Github, Visual Studio Code: Quản lý dự án, lưu, chia sẻ và soạn thảo mã nguồn.
\end{itemize}

\pagebreak