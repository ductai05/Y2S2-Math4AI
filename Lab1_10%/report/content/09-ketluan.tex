\newpage
\section{Kết luận}
\subsection{Hiệu suất các mô hình}

\begin{table}[H]
\centering
\label{tab:metrics}
\begin{tabular}{@{}ccccccccc@{}}
\toprule
\multirow{2}{*}{\textbf{Mô hình}} &
  \multicolumn{8}{c}{\textbf{Phương pháp đánh giá}} \\ \cmidrule(l){2-9} 
 &
  \multicolumn{4}{c|}{\textbf{Tập huấn luyện}} &
  \multicolumn{4}{c}{\textbf{Tập kiểm chứng}} \\ \midrule
\textbf{Tên} &
  \textbf{MSE} &
  \textbf{RMSE} &
  \textbf{MAE} &
  \multicolumn{1}{c|}{$\bm{R^2}$} &
  \textbf{MSE} &
  \textbf{RMSE} &
  \textbf{MAE} &
  $\bm{R^2}$ \\ \midrule
Simple &
  $5.32 \times 10^{10}$ &
  $2.30 \times 10^{5}$ &
  $7.55 \times 10^{4}$ &
  \multicolumn{1}{c|}{0.99} &
  $3.16 \times 10^{11}$ &
  $5.62 \times 10^{5}$ &
  $2.98 \times 10^{5}$ &
  0.930 \\
Multiple &
  $1.80 \times 10^{11}$ &
  $4.25 \times 10^{5}$ &
  $2.41 \times 10^{5}$ &
  \multicolumn{1}{c|}{0.96} &
  $2.20 \times 10^{11}$ &
  $4.69 \times 10^{5}$ &
  $2.98 \times 10^{5}$ &
  0.951 \\
Poly &
  $4.03 \times 10^{10}$ &
  $2.00 \times 10^{5}$ &
  $1.12 \times 10^{5}$ &
  \multicolumn{1}{c|}{0.99} &
  $1.89 \times 10^{11}$ &
  $4.35 \times 10^{5}$ &
  $2.28 \times 10^{5}$ &
  0.960 \\
PCA &
  $5.26 \times 10^{10}$ &
  $2.29 \times 10^{5}$ &
  $9.65 \times 10^{4}$ &
  \multicolumn{1}{c|}{0.99} &
  $2.95 \times 10^{11}$ &
  $5.43 \times 10^{5}$ &
  $2.91 \times 10^{5}$ &
  0.934 \\ \bottomrule
\end{tabular}
\caption{Các đánh giá cho từng mô hình}
\end{table}

\paragraph{Nhận xét:}
\begin{itemize}
    \item Mô hình Simple đạt $R^2$ cao do sự tương quan lớn giữa đặc trưng \texttt{Model} và \texttt{Price}.
    \item Mô hình Multiple cho hiệu suất không chênh lệch nhiều\footnote{Mô hình Multiple có sử dụng Cross-validation trong quá trình huấn luyện} trên tập huấn luyện và kiểm chứng. 
    \item Mô hình Polynomial cho hiệu suất tốt nhất trên cả tập huấn luyện và tập kiểm chứng.
    \item Mô hình PCA đã cải thiện hiệu suất hơn so với mô hình Simple.
\end{itemize}

\subsection{So sánh các loại mô hình}

\begin{table}[H]
\centering
\begin{tabular}{|p{2cm}|p{4cm}|p{5cm}|p{5cm}|}
\hline
\textbf{Đặc điểm} & \textbf{Hồi quy đơn giản} & \textbf{Hồi quy đa biến} & \textbf{Hồi quy đa thức} \\
\hline
Số biến độc lập & Một biến & Nhiều biến & Một biến/Nhiều biến có bậc \\
\hline
Mô hình toán học & $Y = b + \theta_1 X$ & $Y = b + \theta_1 x_1 + \dots + \theta_p x_p$ & $
y = \sum_{|\mathbf{j}| \leq d} \beta_{\mathbf{j}} \mathbf{x}^{\mathbf{j}} + \epsilon
$\\
\hline
Diễn giải hình học & Đường thẳng trong không gian 2 chiều & Siêu phẳng trong không gian $p+1$ chiều & Siêu mặt cong trong không gian nhiều chiều \\
\hline
Ưu điểm & Đơn giản, dễ đánh giá, tránh được một số vấn đề đa cộng tuyến. & Mô hình hóa tuyến tính, đơn giản, ít bị đa cộng tuyến. & Mô hình hóa mối quan hệ phi tuyến tính, dự đoán hiệu quả dữ liệu phức tạp.\\
\hline
Nhược điểm & Hạn chế khả năng mô hình hóa tương tác. & Hạn chế mô hình hóa phi tuyến tính. & Phức tạp, dễ bị cộng đa tuyến, overfitting.\\
\hline
\end{tabular}
\caption{So sánh giữa các mô hình hồi quy tuyến tính}
\end{table}

\paragraph{Nhận xét:}{Hồi quy tuyến tính đa thức vượt trội hơn so với hồi quy tuyến tính đơn giản và hồi quy đa biến trong việc:}

\begin{itemize}
    \item Mô hình hóa các mối quan hệ phi tuyến tính.
    \item Cung cấp khả năng phân tích sự tương tác giữa các đặc trưng của bộ dữ liệu.
    \item Tăng khả năng dự báo chính xác khi có nhiều yếu tố ảnh hưởng, bộ dữ liệu phức tạp.
\end{itemize}

\paragraph{}{Qua thực nghiệm, mô hình hồi quy đa thức (polynomial regression) là mô hình tốt nhất để giải quyết bài toán dự đoán giá xe.}

\pagebreak